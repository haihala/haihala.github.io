\documentclass{article}
\pagenumbering{gobble}  % Removes page number

\usepackage[a4paper, portrait, margin=0.5in]{geometry}
\usepackage[parfill]{parskip}   % Saner paragraph formatting defaults
\usepackage[utf8]{inputenc}     % Default to UTF-8
\usepackage{hyperref}           % Links
\usepackage{collcell}           % Something to do with image + text alignment top of page
\usepackage{etaremune}          % Enumerate backwards
\usepackage{tabularx}           % Dynamically sized table columns
\usepackage{tikz}               % Graphics drawing (skill bars)
\usepackage{arydshln}           % Dashed line dividers
\usepackage{booktabs}           % ???

% Where to find graphics (contains symlinks)
\graphicspath{ {./graphics} }

% Used as less harsh dividers
\setlength{\dashlinedash}{.4pt}
\setlength{\dashlinegap}{2pt}

\newcommand{\cvheader}[2] {
    \begin{tabularx}{\linewidth}{i X}
        \toprule
        #1 & 
        \vspace{1pt} #2\\
    \end{tabularx}
}

% Defines the 'i' column for images (only used in header)
% This helps with centering the image
\newcolumntype{i}{@{\hspace{1ex}}
  >{\collectcell\includepic}l<{\endcollectcell}}

% All of this is basically just to round the image corners a bit
\newcommand{\includepic}[1]{ \raisebox{-\height}{
    \begin{tikzpicture} 
        \begin{scope}
            \clip [rounded corners=.1cm] (0,0) rectangle coordinate (centerpoint) (5,5cm); 
            \node [inner sep=0pt] at (centerpoint) { \includegraphics[width=5cm, height=5cm,keepaspectratio]{#1} }; 
        \end{scope}
    \end{tikzpicture}
}}

\newenvironment{skills}
{
      \tabularx{\linewidth}{c|X} 
      % This is the header row
      Level & Skill list \\
}
{\endtabularx}


% First param is a number for how many skill segments.
% Second param is a list of skills
\newcommand{\skillitem}[2]{
    \hdashline
    \skillbar{#1}&#2\\
}

\newcommand{\skillbar}[1]{
    % Baseline gives it some separation from the top divider
    \begin{tikzpicture}[baseline=3]
        \foreach \x in {0,1,...,2}{
            \ifnum\x<#1
                % Filled box
                \draw[black, very thick, fill=gray] (\x, 0) rectangle (\x+0.8, 0.3);
            \else
                % Empty box
                \draw[black, very thick] (\x, 0) rectangle (\x+0.8, 0.3);
            \fi
        }
    \end{tikzpicture}
}

% Timelines are used for employment and education histories
\newenvironment{Timeline}
  {\tabularx{\linewidth}{Xl} \hdashline}
  {\endtabularx}


% Fields: 
  % Time
  % Title
  % Company / School / explanation
\newcommand{\timelineitem}[3]{
    \textbf{#2} & #1\\
    \footnotesize#3\\
    \hdashline
}
